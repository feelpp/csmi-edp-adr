\documentclass[11pt,utf8]{article}
\usepackage[utf8]{inputenc}
\usepackage[francais]{babel}
\usepackage{graphicx}
%\usepackage[latin1]{inputenc}
%\usepackage[french]{babel}
\usepackage{amsmath,amssymb,amsfonts}
\usepackage{hyperref}

\usepackage{colortbl,booktabs}
\usepackage{listings}
\usepackage{pgfplots,pgfplotstable}
\usepackage{xspace}
\usepackage{tikz}
\usetikzlibrary{arrows,patterns,plotmarks,shapes,snakes,er,3d,automata,backgrounds,topaths,trees,petri,mindmap}


\definecolor{lbcolor}{rgb}{0.95,0.95,0.95}
\definecolor{cblue}{rgb}{0.,0.0,0.6}

\newcommand{\Air}{\text{\textsc{Air}}\xspace}
\newcommand{\PCB}{\text{\textsc{PCB}}\xspace}
\newcommand{\IC}{\text{\textsc{IC}}\xspace}
\newcommand{\ICs}{\text{\textsc{ICs}}\xspace}

\newcommand{\OPUS}{\textsc{OPUS}\xspace}
\newcommand{\FF}{\textsc{Freefem++}\xspace}
\newcommand{\COMSOL}{\textsc{Comsol}\xspace}
\newcommand{\COMSOLUJF}{\textsc{Comsol(UJF)}\xspace}
\newcommand{\COMSOLEADS}{\textsc{Comsol(EADS)}\xspace}
\newcommand{\Feelpp}{\textsc{Feel++}\xspace}
\newcommand{\GMSH}{\textsc{Gmsh}\xspace}
\newcommand{\BOOST}{\textsc{Boost}\xspace}

\newcommand{\UJF}{\textsc{UJF}\xspace}
\newcommand{\EADS}{\textsc{EADS}\xspace}

\newcommand{\vect}[1]{\ensuremath{\mathbf{#1}}\xspace}


\definecolor{Lightgray}{rgb}{0.85, 0.85, 0.85}

\usepackage{filecontents,listings}
\lstset{language=c++,showspaces=false,showstringspaces=false,captionpos=t,literate={>>}{\ensuremath{>>}}1,mathescape}
\lstset{basicstyle=\small\bf\ttfamily}
\lstset{lineskip=-2pt}

%\lstset{emph={inline},emphstyle=\color{red}\bfseries}

\definecolor{cgreen}{rgb}{0.,0.6,0.0}

\definecolor{violet}{rgb}{0.5,0,0.5}
\definecolor{vertfonce}{rgb}{0.,0.5,0.}
\definecolor{rouge}{rgb}{0.5,0.,0.}
\definecolor{bleu}{rgb}{0.,0.,1}
\definecolor{orange}{rgb}{1,0.5,0}
\lstset{keywordstyle=\color{red}\bfseries}
\lstset{
emph={form,form1,form2,integrate,on,grad,gradt,dot,id,dx,dy,dz,idt,dxt,dyt,dzt,idv,dxv,dy,dzv,gradv,div,divt,divv,dn,jump,trans,vec,cst,
  project,P,Px,Py,Pz,one,oneX,oneY,oneZ,hFace,N,Nx,Ny,Nz,sin,cos,min,max,abs,pow,chi,exp,LinearForm,BilinearForm,MixedLinearForm,MixedBilinearForm,
  FESpace,MixedFESpace,integrate,project,addCL},emphstyle=\color{bleu},
emph={[2]\_space,\_range,\_expr,\_quad,\_quad1,\_test,\_trial,\_matrix,\_vector,\_solution,\_element,\_rhs,\_rowstart,\_colstart,\_block,
  \_pattern,\_pattern\_block,\_domainSpace,\_imageSpace,\_mesh,\_name,\_partitions,\_worldcomm,\_worldscomm,
  \_type,\_marker1,\_marker2,\_marker3,\_marker4,\_marker5,\_marker6,\_marker7,\_marker8,\_markerAll,\_argc,\_argv,\_desc},emphstyle={[2]\color{violet}\bfseries},
emph={[3]elements,markerName,boundaryfaces,markedfaces,\_Q,solve, newMatrix,newVector,newZeroMatrix,newZeroVector,
  newBlockMatrix,newBlockVector,localize,element,apply,createMesh,localSize,subWorldComm,transpose,setMarker,add,save,
  dirichlet\_vec,neumann\_scal,updateTime,exportResults,updateBdf,init,FluidMechanics}, emphstyle={[3]\color{rouge}\bfseries},
emph={[4]Backend,BlocksVector,BlocksSparseMatrix,BlocksStencilPattern,Blocks,opInterpolation,FunctionSpace,Lagrange,bases,Pch,unitSquare,WorldComm,
  Mesh,Simplex,Node,Rectangle,Quadrangle,Circle,AppliManagement,Environment,feel\_options,exporter},emphstyle={[4]\color{orange}\bfseries},
emph={[5]size\_type,uint16\_type},emphstyle={[5]\color{red}\bfseries},
emph={[6]GeoTool,vf,cl},emphstyle={[6]\color{cyan}\bfseries}
}


\lstset{commentstyle=\ttfamily\color{cgreen}}
\lstset{numberstyle=\tiny}
\lstset{backgroundcolor=\color{lbcolor},rulecolor=}

\lstset{frame=single,framerule=0.5pt}
\lstset{belowskip=\smallskipamount}
\lstset{aboveskip=\smallskipamount}
\lstset{includerangemarker=false,rangeprefix=\/\/\#\ ,% curly left brace plus space
  rangesuffix=\ \#}


%--------dimension
\textwidth 160 true mm
\textheight 220 truemm
\voffset -15 true mm
\hoffset -18 true mm

\def \x  {\textbf x}
\def \n  {\textbf n}
\def \cT {{\cal T}}


\title{Vérification d'estimation a priori pour le Laplacien}
\author{Méthodes numériques pour les EDP}
\date{}
\begin{document}

\maketitle

\section{Vérification}
\label{sec:verification-1}


\subsection{Le Laplacien}
\label{sec:le-laplacian}


L'objectif est de vérifier que les différentes formulations du Laplacien en
fonction des conditions aux limites (voir chapitre 3 sur les problèmes
coercifs)
\begin{itemize}
\item Dirichlet non homogène
\item Neumann non homogene
\item Fourier (ou encore appelée Robin)
\end{itemize}
satisfont les résultats du théorème 1 page 38 des notes de cours pour des approximations $\mathbb{P}_1$ et
$\mathbb{P}_3$ ($k=1$ et $k=3$) en 2D.

Afin de vérifier ces résultats, nous allons d'abord considérer le problème suivant:
soit $\Omega$ le cercle unité tel que $\partial \Omega = \Gamma_D \cup
\Gamma_N \cup \Gamma_R$ avec
\begin{itemize}
\item $\Gamma_D = \{\mathbf{x} = (\cos\theta,\sin\theta),\ \theta \in (0,\pi/2) \}$
\item $\Gamma_N = \{\mathbf{x} = (\cos\theta,\sin\theta),\ \theta \in (\pi/2,\pi) \}$
\item $\Gamma_F = \{\mathbf{x} = (\cos\theta,\sin\theta),\ \theta \in (\pi,2\pi) \}$
\end{itemize}

\begin{equation}
  \label{eq:3}
  \begin{split}
    -\Delta u &= f\\
    u &= g\ \mbox{ sur }\ \Gamma_D\\
    \frac{\partial u}{\partial n} &= m\ \mbox{ sur }\ \Gamma_N\\
    u + \frac{\partial u}{\partial n} &= l\ \mbox{ sur }\ \Gamma_F\\
  \end{split}
\end{equation}

\begin{enumerate}
\item Créer le maillage d'un domaine $\Omega$ quelconque (pas un
  carré),  nommer les bords physiques
  \texttt{Dirichlet}, \texttt{Neumann} et \texttt{Fourier}.
\item  prendre
  $f=1$, $g=m=l=0$ et $h=0.05$
  \begin{itemize}
  \item écrire la formulation variationnelle
%%  \item écrire le code Feel++
  \item résoudre le problème avec \verb+feelpp_qs_laplacian+ et afficher dans le rapport le maillage ainsi que
    la solution $u$ du problème~(\ref{eq:3})
  \end{itemize}
  \item Ensuite considérez la fonction $u(x,y)=\sin(\pi x) cos(\pi y)$ et calculez
$f,g,m,l$ pour que $u$ soit solution de (\ref{eq:3}). Faites une étude de
convergence en échelle log-log et observer les pentes des droites associées
aux approximations $\mathbb{P}_2$. Sont elles celles
attendues par le théorème 1 page 38?

\end{enumerate}


Afin de présenter les résultats, vous rentrerez les erreurs $L_2$ et $H_1$
dans un tableau\ref{tab:1} et sur une figure~\ref{fig:res}. Les résultats
factices sont dans \texttt{res.dat}.

  \begin{table}[h]
    \centering
    \pgfplotstableread{res.dat}\loadedtable
    \pgfplotstabletypeset[columns={h,error1,error2},
    columns/{h}/.style={
    column type=r,fixed, fixed zerofill,precision=3
    },
    columns/{error1}/.style={
    column name=$\|\cdot\|_{L_2}$,
    sci,sci zerofill,
    precision=2},
    columns/{error2}/.style={
    column name=$\|\cdot\|_{H_1}$,
    sci,sci zerofill,
    precision=2},
    every head row/.style={before row=\toprule,after row=\midrule},
    every last row/.style={after row=\bottomrule}
    ]\loadedtable
    \caption{Erreur de convergence}
    \label{tab:1}
  \end{table}
  \begin{figure}[h]
    \centering
    \begin{tikzpicture}[scale=0.70]
      \begin{loglogaxis}[%x=3cm,
        xlabel=h,ylabel=$\|u-u_h\|$,
        % title={ error curves },
        legend style={at={(0,1)}, anchor=north west}]
        \addplot table[x=h,y={create col/linear regression={y=error1}}]{res.dat};
        \xdef\slopea{\pgfplotstableregressiona}
        \addlegendentry{$\mathbb{P}_1$ pente = $\pgfmathprintnumber{2.17}$}
        \addplot table[x=h,y={create col/linear regression={y=error2}}]{res.dat};
        \xdef\slopeb{\pgfplotstableregressiona}
        \addlegendentry{$\mathbb{P}_2$ pente = $\pgfmathprintnumber{4.36}$}
      \end{loglogaxis}
    \end{tikzpicture}
    \caption{Illustration}
    \label{fig:res}
  \end{figure}


\subsection{Fonctions peu régulières}
\label{sec:fonct-peu-regul}

Solutions singulières. On considere à présent le domaine $\Omega=\left\{\mathbf{x}=r(\cos \theta, \sin \theta)^{T}, r \in\right.$ $\left.(0,1), \theta \in\left(0, \frac{3 \pi}{2}\right)\right\}$ et le problème de Poisson avec condition de Dirichlet
  \begin{equation}
  \label{eq:1}
  \begin{split}
    -\Delta u &= 0\ \mbox{dans}\ \Omega\\
    u &= g\ \mbox{sur} \partial\Omega
  \end{split}
\end{equation}
avec $g$ définie telle que $u$ définie en coordonnée polaire par
$$
u(r, \theta)=r^{2 / 3} \sin \left(\frac{2}{3} \theta\right)
$$
soit la solution exacte du problème. Cette fonction est peu régulière proche de l'origine et conduit donc à des approximations peu précises près de l'origine.
\begin{itemize}
\item Vérifier que le Laplacien de $u$ est bien nul puis déterminer $g$ telle que $u$ soit solution du problime.
\item Montrer que $u \in H^{1}(\Omega)$ mais que le gradient de $u$ n'est pas borné à l'origine. En déduire que
$$
u \notin H^{2}(\Omega)
$$
\item  Créer le maillage avec Gmsh.
\item Faire l'étude de convergence. Qu'observez vous?
\end{itemize}
\end{document}
